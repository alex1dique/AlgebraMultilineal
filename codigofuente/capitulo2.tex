\setlength{\parskip}{\baselineskip}% Para que me genere el espacio entre parrafos dejando una line en blanco entre ellos.

\chapter{Formas y Operadores}
	\section{Formas Bilineales}
		\begin{defi} \label{def2.1}
			\normalfont
			Sean  $ U $ ,  $ V $  y  $ W $  espacios vectoriales sobre un campo  $ K $ .
			Una función $ f:U $ x $ V $ $ \rightarrow W $ se llama bilineal si:
			
			$ (i) f(u_{1} + u_{2}, v) = f(u_{1}, v) + f(u_{2}, v) $ \\
			$ (ii) f(u, v_{1} + v_{2}) = f(u, v_{1}) + f(u, v_{2})  $ y \\
			$ (iii) f(\lambda u, v) = \lambda f(u,v) = f(u, \lambda v); u_{1},u_{2},u \in U; v_{1}, v_{2}, v \in V; \lambda \in K.$  
			
			Es decir, $ f:U $  x $V \rightarrow W $ es bilineal si es lineal en cada variable cuando la otra se mantiene fija.

		\end{defi}
	
		\textbf{Observación}. Lo anterior significa que si $ v \in V $ se mantiene fija en $ U $ x $ V $ , la función $ u \longmapsto f(u, v) $ es lineal y por lo tanto es un elemento de $ Hom_{K}(U, W) $. De igual forma, si $ u \in U $ se mantiene fija en $ U $ x $ V $ , la función$  v \l f(u, v) $ es lineal y pertenece a $ Hom_{K}(V, W) $. Esto no significa que f sea lineal como función $ f:U $  x $V \rightarrow W $. Por ejemplo, $ f: {\rm I\!R}$ x ${\rm I\!R} \rightarrow {\rm I\!R} $ dada por $ f(u, v) = uv $ es bilineal pero no es lineal. Otro ejemplo, $ f: {\rm I\!R} $ x ${\rm I\!R} \rightarrow {\rm I\!R} $ dada por $ f(u, v) = u+v $ es lineal pero no es bilineal. 

		\begin{ejem}
			\normalfont
			Sea $ A $ una matriz de $ m $ x $ n $. Definamos
			\[ f:K^{m} \times K^{n} \rightarrow K \]
			
			mediante $ f(X,Y) =$ $^{t}XAY$. esto es
			\[ \begin{matrix}
				\left(x_{1}, \ldots, x_{m}\right)\left(\begin{matrix}
				a_{11} &\ldots& a_{1n} \\ \vdots & &  \vdots \\ a_{m1} & \ldots & a_{mn}
				\end{matrix}\right)\left(\begin{array}{c}
					y_{1}\\ \vdots \\ y_{n}
				\end{array}\right)
				\\ \\
				= \left(\sum_{i=1}^{m}x_{i}a_{i1},\ldots,\sum_{i=1}^{m}x_{i}a_{in}\right) \left(\begin{array}{c}
				y_{1}\\ \vdots \\ y_{n}
				\end{array}\right)
				\\
				 = \sum_{j=1}^{n} \sum_{i=1}^{m}x_{i}a_{ij}y_{j} 
				 \\ \\
				 = \sum_{j=1}^{n} \sum_{i=1}^{m}a_{ij}x_{i}y_{j}. 
			\end{matrix} 
			 \]
		\end{ejem}
			Es inmediato comprobar que $ f(X,Y) \in K $ y que es bilineal, ya que las propiedades de las matrices establecen que $ ^{t}XA(Y+Y') = ^{t}XAY+^{t}XAY'$ y $ ^{t}XA(\lambda Y) = \lambda(^{t}XAY) $.
			
			Por ejemplo, si $ A = \left(\begin{matrix}
			2 & 1 & 1 \\ 1 & 3 & 3 \\ 2 & 1 & 1 
			\end{matrix}\right)$, $ X = \left(\begin{matrix}
			x_{1} \\ x_{2} \\ x_{3}
			\end{matrix}\right) $ y $ Y = \left(\begin{matrix}
			y_{1} \\ y_{2} \\ y_{3}
			\end{matrix}\right) $ entonces 
		
			$f(X,Y) = (x_{1}, x_{2}, x_{3})\left(\begin{matrix}
			2 & 1 & 1 \\ 1 & 3 & 3 \\ 2 & 1 & 1 
			\end{matrix}\right)\left(\begin{matrix}
				y_{1} \\ y_{2} \\ y_{3}
			\end{matrix}\right)	
			$
			
			 \textcolor{white}{\textit{f(X,Y)=}}$=(2x_{1} + x_{2} + 2x_{3}, x_{1} + 3x_{2} + x_{3}, x_{1} + 3x_{2} + x_{3})\left(\begin{matrix}
			 y_{1} \\ y_{2} \\ y_{3}
			 \end{matrix}\right)
			$
		
			\textcolor{white}{\textit{f(X,Y)=}}$ =2x_{1}y_{1} + x_{2}y_{1} + 2x_{3}y_{1} + x_{1}y_{2} + 3x_{2}y_{2} + x3y_{2} + x_{1}y_{3} + 3x_{2}y_{3} + x_{3}y_{3}.| $
			
			Si en \ref{def2.1}, $ W = K $ diremos que $ f $ es una \textit{forma bilineal}. Denotamos con $ L^{2}(U, V ;K) $ el conjunto de formas bilineales de $ U \times V $ en $ K $. Si $ U = V $, simplemente denotamos a $ L^{2}(V, V ;K) $ con $ L^{2}(V ;K) $ o con $ Bil(V) $ entendiéndose que
			se trata de formas bilineales de $ V \times V $ en $ K $, que son las que consideraremos en adelante.
			
			A $ Bil(V) $ le podemos dar una estructura de espacio vectorial mediante las reglas \\
		$ (f+g)(u,v) = f(u,v) + g(u,v) $ y $(\lambda f)(u,v) = \lambda f(u, v)$ para $f, g \in Bil(V), \lambda \in K. $
		
		Consideremos el caso en que tengamos el espacio vectorial de homomorfismos $ Hom_{K}(V,K $. Sus elementos $ f:V \rightarrow K, $ que son homomorfismos o aplicaciones lineales, se acostumbra llamarlos \textit{funcionales lineales} o \textit{formas lineales}. También se acostumbra denotar a $ Hom_{K}(V,K $ con $ L^{1}(V;K) $ o simplemente V* y se le llama espacio dual de V. Su estructura esta dada por
		
		\begin{center}
			$ (f + g)(v) = f(v) + g(v)$  y 
			
		$ (\lambda f)(v) = \lambda f(v); v \in V,\lambda \in K. $
		\end{center}
		
		Aquí vemos a $ K $ como espacio vectorial sobre si mismo.
		
		\begin{ejem}
			\normalfont
			Sea $ V = M_{n}(K) $ el espacio de las matrices cuadradas de $ n \times n $. \\ Sea $ f = tr:M_{n}(K) \rightarrow K $ dada por $ tr(A) = a_{11}+a_{22}+\ldots+a_{nn},$ i.e., la traza de \\ la matriz  $ A = \left(\begin{matrix}
			a_{11} & \ldots & a_{1n} \\ \vdots & & \vdots \\ a_{n1} & \ldots & a_{nn} 
			\end{matrix}\right) $. Como $ tr(A+B) = trA+trB $ y $ tr(\lambda A) = \lambda trA, $ \\ $ tr $ es un funcional.
		\end{ejem}
		