\setlength{\parskip}{\baselineskip}% Para que me genere el espacio entre parrafos dejando una line en blanco entre ellos.

\chapter{Formas y Operadores}
	\section{Formas Bilineales}
		\begin{defi}
			\normalfont
			Sean  $ U $ ,  $ V $  y  $ W $  espacios vectoriales sobre un campo  $ K $ .
			Una función $ f:U $ x $ V $ $ \rightarrow W $ se llama bilineal si:
			
			$ (i) f(u_{1} + u_{2}, v) = f(u_{1}, v) + f(u_{2}, v) $ \\
			$ (ii) f(u, v_{1} + v_{2}) = f(u, v_{1}) + f(u, v_{2})  $ y \\
			$ (iii) f(\lambda u, v) = \lambda f(u,v) = f(u, \lambda v); u_{1},u_{2},u \in U; v_{1}, v_{2}, v \in V; \lambda \in K.$  
			
			Es decir, $ f:U $  x $V \rightarrow W $ es bilineal si es lineal en cada variable cuando la otra se mantiene fija.

		\end{defi}
	
		\textbf{Observación}. Lo anterior significa que si $ v \in V $ se mantiene fija en $ U $ x $ V $ , la función $ u \longmapsto f(u, v) $ es lineal y por lo tanto es un elemento de $ Hom_{K}(U, W) $. De igual forma, si $ u \in U $ se mantiene fija en $ U $ x $ V $ , la función$  v \l f(u, v) $ es lineal y pertenece a $ Hom_{K}(V, W) $. Esto no significa que f sea lineal como función $ f:U $  x $V \rightarrow W $. Por ejemplo, $ f: {\rm I\!R}$ x ${\rm I\!R} \rightarrow {\rm I\!R} $ dada por $ f(u, v) = uv $ es bilineal pero no es lineal. Otro ejemplo, $ f: {\rm I\!R} $ x ${\rm I\!R} \rightarrow {\rm I\!R} $ dada por $ f(u, v) = u+v $ es lineal pero no es bilineal. 

		\begin{ejem}
			\normalfont
			Sea $ A $ una matriz de $ m $ x $ n $. Definamos
			\[ f:K^{m} \times K^{n} \rightarrow K \]
			
			mediante $ f(X,Y) =$ $^{t}XAY$. esto es
			\[ \begin{array}{c}
				\left(x_{1}, \ldots, x_{m}\right)\left(\begin{matrix}
				a_{11} &\ldots& a_{1n} \\ \vdots & &  \vdots \\ a_{m1} & \ldots & a_{mn}
				\end{matrix}\right)\left(\begin{array}{c}
					y_{1}\\ \vdots \\ y_{n}
				\end{array}\right)
				\\ \\
				= \left(\sum_{i=1}^{m}x_{i}a_{i1},\ldots,\sum_{i=1}^{m}x_{i}a_{in}\right) \left(\begin{array}{c}
				y_{1}\\ \vdots \\ y_{n}
				\end{array}\right)
			\end{array} \]
		
			\[  = \sum_{j=1}^{n} \sum_{i=1}^{m}x_{i}a_{ij}y_{j} \]
			 \[ = \sum_{j=1}^{n} \sum_{i=1}^{m}a_{ij}x_{i}y_{j}.  \]
			
		\end{ejem}