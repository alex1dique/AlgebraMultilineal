%comandos
\newtheorem{axi}{\sc Axioma}%Axiomas
\renewcommand{\theequation}{\arabic{equation}}
%-----------------------
\chapter{ESPACIOS DE HILBERT}

{\Large \bf Resumen}\\

En este trabajo se estudian los axiomas para la formación de un espacio de Hilbert de naturaleza estadística definiendo los elementos pertenecientes al espacio como funciones, variables estocásticas u procesos estocásticos, Se completan las definiciones de la operación interna definida en los axiomas que, en el caso de ser funciones, éstas deben ser de cuadrado integrables; y en el de variables estocásticas, el axioma del producto interno es una esperanza matemática.
\\

No se estudian los procesos estocásticos, pero se comprende fácilmente que los elementos son procesos y el producto interno son esperanzas matemáticas coincidiendo con las funciones de covarianza.\\

Si los elementos son funciones, toda la Teoría de las Funciones Ortogonales puede desarrollarse y una función de densidad puede converger en media cuadrática en desarrollos conocidos que se deducen como casos particulares.
\\

Si los elementos son de naturaleza estocástica, son un conjunto de variables estocásticas, puede construirse una base aleatoria ortogonalizada y toda variable distinta puede desarrollarse calculando sus coeficientes regresores que son las coordenadas de los elementos aleatorios ortogonalizados. Se estudian el hiperplano, recta y plano de regresión como casos particulares. \\

Finalmente, la distancia al cuadrado de la variable regresora sobre el hiperplano de regresión, no es sino la varianza residual y que es nula cuando la variable regresora sea un punto del hiperplano. En este caso la dimensionalidad es la del subespacio euclídeo.
\\

Palabras clave: Espacio de Hilbert: funciones ortogonales en el espacio de Hilbert; espacio estocástico de Hilbert.
\\

{\Large \bf Introducción}\\

En este articulo expondremos brevemente el concepto de espacio de Hilbert y su fundamentación axiomática para dedicarnos, en principio, a las aplicaciones estadísticas en sus dos aspectos: cuando los elementos correspondientes son de naturaleza funcional a estocástica.
\\

En las aplicaciones funcionales trato, en principio, de las funciones ortogonales en
general y especialmente dedico atención preferente a las funciones de Fuurier para,
entre otros, demostrar la relación entre la función característica y su función de
densidad.
\\

Siguiendo el mismo método de las funciones ortogonales (generalizadas respecto a una función ponderativa o núcleo) estudiaremos, a título de ejemplo, algunos desarrollos sencillos, como son los de Charlier y de Hermite.
\\

De forma semejante podrían construirse polinomios ortogonales, como son los de
Legendre, Thebychev, Laguerre, Hermite, etc., que comentaremos brevemente.
La segunda parte de nuestro estudio, preferentemente la dedicamos al espacio
estocástico de Hilbert. En esta parte puntualizaremos la métrica de este espacio que,
como se verá, son esperanzas matemáticas.
\\

Es importante la formación de elementos estocásticos ortonormalizados; es decir,
construir una base y cualquier elemento de naturaleza estocástica representarlo en esta
base. La determinación de las coordenadas en la base elegida es similar a la de
naturaleza funcional. Consecuencia inmediata es la posibilidad de representar una recta
o un plano de regresión por medio de la base estocástica ortonormal izados elegida, que
tiene la propiedad de ser mínima la varianza residual.
\\

\section{DEFINICIÓN Y AXIOMAS COMPLEMENTARIOS DE HILBERT}

\subparagraph{Definición:}
Un espacio con producto interno (o espacio  pre Hilbert) es un e.v  $ X $ con un producto interno definido en $X$.
\subparagraph{Proposición:} Si la norma es producida por un producto interno $y$ $x$, $ y \in x$. Entonces se cumple la regla del paralelogramo: 

			 \[ \parallel x+y \parallel^{2}  + \parallel x-y \parallel^{2} = 2(  \parallel x \parallel^{2} + \parallel y \parallel^{2}  ) \] 
			 $\underline{\underline{p_{a}}}$ \\			 
			\[ \parallel x+y\parallel ^{2} = (\sqrt{<x+y , x+y>})^{2} \]
			\[ = {<x+y , x+y>} \]
			\[ = {<x, x+y> +<y,  x+y>} \]
			\[ =<x, x>+<x, y>+<y, x>+<y, y> \]
			\[ =\parallel x \parallel ^{2}+<x, y>+<y, x>+\parallel y \parallel ^{2} \]
			
			\[ \parallel x-y\parallel ^{2} = (\sqrt{<x-y , x-y>})^{2} \]
			\[ = {<x-y , x-y>} \]
			\[ = {<x, x-y> +<y,  x-y>} \]
			\[ =<x, x>-<x, y>-(<y, x>-<y, y>) \]
			\[ =\parallel x \parallel ^{2}-<x, y>-<y, x>+\parallel y \parallel ^{2} \]
			
			\[ \parallel x+y \parallel^{2}  + \parallel x-y \parallel^{2} = 2 \parallel x \parallel^{2} + 2\parallel y \parallel^{2}  \] 
				 \[ = 2( \parallel x \parallel^{2} + \parallel y \parallel^{2}) \] 
				 
		En $\mathbb{R} $ pI es $(a, b)(c, d) = ac + bd$\\
		$\parallel x \parallel = \sqrt{<x, x>}$\\
		$\parallel x \parallel^{2}= <x, x>$\\
		En $\mathbb{R}$ $<x, y>$ = $<y, x>$
	\subparagraph{Proposición:}  Si $X$ es un espacio con producto interno, entonces:\\
a) $ <\alpha x + \beta y, z> = \alpha <x,z> + \beta <y, z>$\\
b) $<x, \alpha y> = \bar{\alpha} <x, y>$\\
c) $<x, \alpha y + \beta z> = \bar{\alpha}<x, y> + \bar{\beta}<x, z>$ para cada $x, y, z$  $\in x$ y $\alpha_{1} \beta \in{\mathbb{K}}$  \\

$\underline{\underline{p_{a}}}$ \\

a) $<\alpha x + \beta y, z>= <\alpha x, z>+<\beta y, z>; (Ip1)$

{\setlength{\parindent}{1.28in}= $\alpha<x, z>+\beta<y, z>; (Ip2)$}

b) $<x, \alpha y> = <\overline{\alpha y ,x}> ;$ por $ (Ip3) $

{\setlength{\parindent}{0.9in}= $\overline{\alpha<y, x>} ;$ por $ (Ip2)$}

{\setlength{\parindent}{0.9in}= $\overline{\alpha} . \overline{<y, x>} $}

{\setlength{\parindent}{0.9in}= $\overline{\alpha}  {<y, x>} ; (Ip3)$}

c) $<x, \alpha y + \beta z> = <\overline{\alpha y +\beta z, x}>  (Ip3)$

{\setlength{\parindent}{1.27in}= $\overline{ \alpha< y, x> + \beta<z, x>}; por$}

{\setlength{\parindent}{1.27in}= $\overline{ \alpha< y, x>}+ \overline{\beta<z, x>}$}

{\setlength{\parindent}{1.27in}= $\overline{ \alpha< y, x>}+ \overline{\beta<z, x>}$}

{\setlength{\parindent}{1.27in}= $\overline{ \alpha}< y, x>+ \overline{\beta}<z, x>$}

$\parallel x \parallel = \sqrt{<x, x>} = \sqrt{ {\xi_{1}}^{2}+{\xi_{2}}^{2}+...+{\xi_{n}}^{2} }$ 

$\parallel x-y \parallel = \sqrt{(\xi_{1}-\eta_{1})}^{2}+...+{(\xi_{2}-\eta_{n})}^{2}$\\

El producto interno induce la misma métrica con lo que probamos que era completo, así  $\mathbb{R}$ es un espacio de Hilbert y debemos probar que $<x, y>$ es un peso y así en $\mathbb{R}^{n}$
\begin{center}
	$ <x, y> = \xi_{1} \eta_{1}+\xi_{2} \eta_{2}+...+\xi_{n} \eta_{n} $ en $ \mathbb{R}^{n} $
\end{center} 

Espacio $\not \subset^{n}:$ El espacio $\not \subset^{n}$ es un espacio de Hilbert con producto interno definido por:
\begin{center}
	$ <x, y> = \xi_{1} \eta_{1}+\xi_{2} \overline{\eta_{2}}+...+\xi_{n} \overline{\eta_{n}} $ en $ \not \subset^{n} $
	
$\parallel x \parallel = \sqrt{<x, x>}$

$= \sqrt{\xi_{1}.\overline{\xi_{1}}+...+{\xi_{n}.\overline{\xi_{n}}} }$\\

$ = \sqrt{|\xi_{1}|^{2}+...+|\xi_{n}|^{2}}$\\

$\parallel <x - y> \parallel = \sqrt{|\xi_{1}-\eta_{1}|^{2}+...+|\xi_{n}-\eta_{n}|^{2}}$\\

Luego $\not \subset^{n}$ es Hilbert
\end{center} 

\subsection{Axiomática}

A finales del siglo pasado se estudiaron las propiedades primarias de matrices. Los
métodos de Algebra Lineal estudian los conceptos de espacios y se aplican a sus
transformaciones, formas de operadores, proyectores, etc., aunque, como dice Máltsev (*), el objeto del Algebra Lineal es el estudio de las «matrices, espacios y formas algebraicas, cuyas teorías están estrictamente vinculadas» .
\\

Son inmensas las aplicaciones a la Estadística y a la Econometría y he creída
necesario dedicar estas breves páginas en forma elemental al estudio de los conceptos
básicos de los espacios de Euclides y de Hilbert, para después aplicarlos a la Estadística.\\

En los espacios abstractos se denominan elementos a un conjunto de puntos,
variables estocásticas, vectores, funciones que gozan determinadas propiedades, que
se establecen como axiomas o postulados. Particularmente importantes son los espacios
euclídeo de n dimensiones y el espacio de Hilbert, de infinitas dimensiones. \\

Sintetizaremos la axiomática para ambos espacios, representada por $H$ el espacio
de Hilbert o euclídeo , $n$ dimensiones; pero cuando expresamente nos referimos al
euclídeo, representaremos el espacio por $H_{n}$, indicando el subíndice la dimensionalidad.\\

\begin{axi}[AXIOMA DE GRUPO]
	Dados sus elementos pertenecientes al espacio x, e y E H está definida la operación adición, con las propiedades de poseer el elemento neutro, e1 simétrico (la operación adición es conmutativa y asociativa).
\end{axi}


\begin{axi}[AXIOMA DE LINEALIDAD]
	Para todo $x \in  H$ y para cualquier número completo está definida la operación
	producto $cx$ o $c.x$, que goza con respectó a la adición de ser distributiva a derecha e
	izquierda, y en consecuencia, si $x, y$  $\in{H}$, $c(x + y)$ = $cx + cy$ $\in{H}$ ; $c(dx)$=$(cd)x$  $\in H$, donde c y d son números escalares cualesquiera. EI producto por 0 y 1 son: $0_{x} = 0$,
	$l . x = x$.
\end{axi}

\begin{axi}[AXIOMA MÉTRICO]
	Se define una operación interna (también denominada producto interno) con las siguientes propiedades:
\\
	
	Si $x, y \in{H}$ denominamos producto interno al número compleja $(x, y)$, que cumple
las siguientes condiciones:
\\
		
		
		\[ (ax, y)= a(x, y) \text{a un escalar cualquiera.}\] 
		 \[(x, y ) = ( \overline{y.x} ) \]
        \begin{equation}
        (x + y, z) - (x.z ) + (y, z )  
        \end{equation}      
        \[ (ax + by, z)= a( x, z)+ b( y, z) \text{corolario de las dos precedentes.} \] 
		\[(x, ay )= \overline{a}(x, y) \text{gualmente corolario de las dos primeras, y también}\]
		\[( x, x ) \geq 0\]
		\[\text{Se denomina longitud de x o norma:}\] 
	    \begin{equation}
	    	\parallel x \parallel = (x, x )^{1/2}
	    \end{equation}
		\[\text{y al producto:}\]
		\begin{equation}
			(x-y, x-y )^{1/2} = \parallel x -y \parallel
		\end{equation}
		
		Se llama distancia del elemento $ x $ al elemento $ y $; si es cero, $x=y$ , y mayor que cero cuando $x \not = y$.
\end{axi}

\begin{axi}[AXIOMAS COMPLEMENTARIOS DE HILBERT]
	Axiomas de convergencia. Dos axiomas se introducen en este espacio:\\
	
	a) Propiedad de Cauchy  si  $x_{n} \in H$ y dado un $ \varepsilon > 0 $ por pequeño que sea, si para
	un  n > N se verifica:
	
	\begin{equation}
		\parallel  x_{n}  - x_{n+p} \parallel < \varepsilon 
	\end{equation}
	
	 existe un elemento $x$ al que converge $ x_{n} $
	 
	 \begin{equation}
	 \lim\limits_{n \rightarrow x} \parallel x - x_{n} \parallel = 0
	 \end{equation}
	 Suele sustituirse este axioma por la propiedad denominada espacio completo.\\
	 
	 b) Separabilidad. Cuando hay una sucesión $(x_{i} \in H)$ al que se aproxima a $x \in H$ con error arbitrariamente pequeño, podemos del espacio H extraer una sucesión de elementos de $  x_{n} \in H $ , de forma que:
	 
	 	\begin{equation}
	 \parallel  x  - x_{n} \parallel < \varepsilon 
	 \end{equation}
	 
	 Se denomina a esta propiedad espacio separable cuando el error es arbitrariamente pequeño. 
\end{axi}

\section{ORTOGONALIDAD}

\subparagraph{Definición:} Un elemento $x$ de un espacio con producto interno $x$ se dice que es ortogonal a un elemento $y \in x$ si $<x, y> = 0$ \\

También decimos que $x$ y $y$ son ortogonales, y escribimos $\underbrace{x \perp y}$\\

Similarmente, para subconjuntos $A, B \subseteq X$ escribimos $x \perp A$ si $x \perp a$ para todo $a \in A$ y todo $b \in B$.
\subparagraph{Ejemplos:}
Espacio Euclidiano  $ \mathbb{R}^{n} $: El espacio euclideano  $ \mathbb{R}$ es un espacio de Hilbert con producto interno definido por:

		$ <x, y> = \xi_{1} \eta_{1} + \xi_{2} \eta_{2} +  ... + \xi_{n} \eta_{n} $\\
		donde\\
		$x = (\xi_{1}, ..., \xi_{n}), y = (\eta_{1}, \eta_{2}, ..., \eta_{n}) \in \mathbb{R}^{n} $\\
		
		\setcounter{equation}{0}
%reiniciar numeracion de las ecuaciones
		
	\subsection{CONCEPTOS DE ORTOGONALIDAD}
	
	Dados dos elementos $ x, y \in H $ se denominan ortogonales cuando: 
	\begin{equation}
	(x, y) = 0                                                                 
	\end{equation} 
	La ortogonalidad se representa también de la forma:
	\begin{equation}
		x \perp y
	\end{equation}   
	En el caso particular, si para $ x \in H $, el producto escalar es:
	\begin{equation}
		(x, x) = 1
	\end{equation}
	se denomina al elemento $ x $ que está normalizado o tiene módulo unidad.\\
	
	En un conjunto de puntos: $ e_{h} \in H_{n} $ $ (h = 1, 2, 3, ..., n) $ que poseen las propiedades (1) y (3).
	

	\begin{equation}
		\begin{array}{lccl}
				& h = k            & \delta_{hh} = 1\\
				(e_{h}, e_{k}) \delta_{hk}   \\
				&  h \not = k  & \delta_{hk} = 0        
		\end{array}	 
	\end{equation}
	donde $ \delta_{hk} $ es el símbolo de Kronocker.\\
	Tal conjunto de puntos$  e_{h} $ se denomina: elementos ortonormalizados.
	En una sucesión $ x_{1}, x_{2}, ..., x_{n} $ de elementos de $ H $(o de $ H_{n} $) que cumplen la (1) por los axiomas (1) de 2.1.1, su producto interno es:
	
	\begin{equation}
		\parallel x_{1}+x_{2}+x_{3}+\ldots+x_{n} \parallel^{2}  = 	\parallel x_{1} \parallel^{2}+\parallel x_{2} \parallel^{2}+\parallel x_{3} \parallel^{2}+\ldots+\parallel x_{n} \parallel^{2}
	\end{equation}
	Si el espacio es de Hilbert, la expresión (5) \\
	\begin{equation}
		\parallel x_{2} \parallel^{2} = \lim\limits_{n \rightarrow \infty} \sum_{i = 1}^{n} |x_{i}|^{2}
	\end{equation}
	Cuando x es ortogonal a todo elemento de un subconjunto$ L $ de$ H $ decimos que es ortogonal a $L$ y escribirnos $x \perp L$. Si $ L_{1} $ y $ L_{2} $ son dos subconjuntos, tales que todo elemento de $ L_{1} $, es ortogonal a todo elemento de $ L_{2} $, diremos que ambos subconjuntos son ortogonales si cumplen (1) para todos los elementos de los subconjuntos correspondientes.
	
\section{COORDENADAS EN LOS ESPACIOS Hn Y H}

Sea el conjunto (4) de 2.2 de elementos ortonormalizados, pudiendo ser n elementos (espacio euclideo) o un conjunto numerable (espacio de Hilbert).\\

Todo punto $ x \in H_{n} $ (o $x \in H$) es representable por una combinación lineal.
	\setcounter{equation}{0}%reiniciar numeracion de las ecuaciones
\begin{equation}
		\begin{array}{lcccl}
		x = & x_{1}e_{1}+x_{2}e_{2}+x_{3}e_{3}+\ldots+x_{n}e_{n} & & & x_{i} \in C
		\end{array}
\end{equation}
\begin{equation}
\text{donde } e_{1}  \text{ son los elementos básicos (4) de 2.2 ortonormalizados. } \text{Si } n \text{ tiende a infinito, el punto } x \in H.
\end{equation}
Multiplicando escalarmente la (1) por $ e_{i} $ tenemos:
\begin{equation}
(x, e_{i}) = x_{i}
\end{equation}
recordando las (4) de 2.2.\\

A $ x_{i} $ se denomina la coordenada $ x_{i} $, y $ x $ puede representarse en función de sus
coordenadas por la (1) o una sucesión ordenada:
\begin{equation}
x = (x_{1}, x_{2}, \ldots, x_{i}, \ldots, x_{n})
\end{equation}
Cuando $ x \in H $, la sucesión de las coordenadas se extiende a infinito.

\section{PROCESO DE NACIONALIZACIÓN DE GRAM-SCHMIDT}

Hemos partido de la hipótesis que conociamos una base en un espacio $ H_{h} $ o $ H $.

A veces se desconoce, y el método de Schmidt se utiliza en los espacios funcionales
para determinar una base.

sea:
	\setcounter{equation}{0}%reiniciar numeracion de las ecuaciones
\begin{equation}
v_{1}, v_{2}, \ldots,v_{n}
\end{equation}
un sistema de elementos pertenecientes a un espacio H.

Se pretende construir un sistema ortonormalizado:
\begin{equation}
(e_{1}, e_{2},  \ldots)
\end{equation}
de forma que el elemento $ e_{j} $ se expresa en función lineal de los elementos $ v_{1}, v_{2}, \ldots,v_{n}, \ldots $ \\

En principio, formemos los elementos $ e_{j} $  auxiliares de la forma siguiente y a partir
de ellos los elementos $ e_{j} $, que son ortonormalizados:
\[ e'_{1} = e_{1} = \frac{v_{1}}{(v_{1}, v_{1})^{1/2}} ;(e_{1}, e_{1}) = 1 \]
\[ e'_{2} = v_{2} - (v_{2}, e_{1})e_{1};  e_{2} = \frac{e'_{2}}{\sqrt{(e'_{2}, e'_{2})}} \]
\[ e'_{3} = v_{3} -(v_{3}, e_{1})e_{1}- (v_{2}, e_{2})e_{2};  e_{3} = \frac{e'_{3}}{\sqrt{(e'_{3}, e'_{3})}} \]
\begin{equation}
e' = v_{n} -\sum_{i = 1}^{n-1} (v_{n}, e_{i})e_{i};  e_{n} = \frac{e'}{\sqrt{(e'_{n}, e'_{n})}} 
\end{equation}

Los vectores $ e_{j} $ son auxiliares y de los $ e_{i} $ se comprueba la ortogonalización de manera sencilla. Para:
\[ (e'_{2}, e_{1}) = (v_{2}, e_{1}) - (v_{2}, e_{1})(e_{1}, e_{1}) = 0\]
por la (3).\\
Supuesto $ j < n $ por inducción,
\begin{equation}
(e'_{n}, e_{j}) = (v_{n}, e_{j}) - \sum (v_{n}, e_{i})(e_{i}, e_{j})
\end{equation}
\[ = (v_{n}, e_{j}) - (v_{n}, e_{j}) = 0 \]

y la normalización de las vectores o elementos $ e_{i} $ es inmediata a partir de los auxiliares.
\section{ DISTANCIA MÍNIMA DE $x \in H$ A UN SUBESPACIO Hn}

El problema planteado es encontrar una descomposición de $ x $ en dos: uno perteneciente al espacio $ H_{n} $ y otro $ y \in H_{n} $, de forma que $ y \perp H_{n} $

	\setcounter{equation}{0}%reiniciar numeracion de las ecuaciones
	
\begin{equation}
x = x' + y
\end{equation}
El elemento $ x' \in H_{n} $ será combinación lineal de los elementos básicos de este
subespacio:
\begin{equation}
x' = \sum_{i = 1}^{n} a_{i}e_{i}
\end{equation}
Resolveremos por mayor sencillez en el caso de ser los elementos $ x \in H $ reales, en
cuyo caso para $ \forall x$, $ y \in H $ será:
\begin{equation}
		(x, y ) = (y, x )
\end{equation}
de acuerdo con el axioma métrico (3) de 2.1.1 \\

La distancia al cuadrado $ \parallel D \parallel ^{2} $ es:
\[ \parallel D \parallel^{2} = \parallel x - x' \parallel^{2} = x - \sum_{i = 1}^{n} a_{i}e_{i}, x - \sum_{i = 1}^{n} a_{j}e_{j} =\] \\
\begin{equation}
		= (x, x) - \sum_{i = 1}^{n} (x_{i}, e_{i})^{2} + \sum_{i = 1}^{n} [(x_{i},e_{i})  - a_{i}]^{2}
\end{equation}
La distancia $\parallel D \parallel$ no depende de los dos primeros sumandos y sí del término no
negativo:
\begin{equation}
	 \sum_{i = 1}^{n} [(x_{i},e_{i})  - a_{i}]^{2}
\end{equation}
\begin{equation}
a_{i} = (x, e_{i})  = x_{i}
\end{equation}
se anulan todos los términos de (5) y el valor mínimo de (4) se reduce:
\begin{equation}
\text{ min }  \parallel D \parallel ^{2} = (x, x) -  \sum_{i = 1}^{n} (x, e_{i}) ^{2}\\
\end{equation}
\[ = (x, x) -  \sum_{i = 1}^{n} x_{i}^{2} \]

recordando la (6).\\
La (1) la escribiremos sustituyendo $ a_{i} $ por los valores determinados por la (6):
\begin{equation}
x' = \sum_{i = 1}^{n}x_{i}e_{i}
\end{equation}
Si las coordenadas se han elegido según la (6) de la (1) y (7) deducimos para $ (y, y) $:
\begin{equation}
(y, y) = (x, x) -  \sum_{i = 1}^{n}x_{i}^{2}
\end{equation}
La distancia $ x $ a su aproximación $ x' $, combinación lineal $ x' \in H_{n} $, puede hacerse tan
pequeña como queramos. Si el conjunto es denso, decimos que x se puede aproximar con expresiones lineales con «error cuadrático» y escribimos:
\begin{equation}
x' =^{2} \sum_{i = 1}^{n}x_{i}e_{i}
\end{equation}
Cuando $ n \rightarrow \infty $ y el error tiende a cero, llamamos convergencia cuadrática a la expresión (10).

\section{FUNCIONES ORTOGONALES EN LOS ESPACIOS DE HILBERT}
	\setcounter{equation}{0}%reiniciar numeracion de las ecuaciones
	
\textbf{GENERALIDADES}

La elección de los elementos básicos pertenecientes a Hilbert pueden ser funciones según dijimos y subrayamos en 2.1.\\

En tal caso, se define previamente el campo de variabilidad $ C =(a, b) $, que
supondremos un intervalo cualquiera.\\

Llamamos sucesión de funciones ortonormalizadas dentro del intervalo $ (a, b) $ a:\\
\begin{equation}
	f_{0}, d_{1}, f_{2}, \ldots, f_{n}
\end{equation}
con las condiciones:
\begin{equation}
\int_{a}^{b} f_{i}f_{j}dx = \delta_{ij} = 
\left\{ \begin{array}{lcl}
	 	1  & i  = j\\
	    0 & i \not = j
\end{array}
\right.
\end{equation}
Definamos la operación:
\begin{equation}
(f, g) = \int_{a}^{b} f \overline{g}dx
\end{equation}
para $\forall f, g$ y examinemos si podemos formar un espacio funcional de Hilbert, con los siguientes axiomas:
\setcounter{axi}{0}%reiniciar numeracion de las ecuaciones
\begin{axi}[AXIOMA DE GRUPO]
	Dadas  dos funciones de (1) cualquiera $ f ,  g \in H$, la suma $ f + g \in H; $ la diferencia  $ f - g \in H; $ existe un elemento neutro y el simétrico que pertenecen a $ H $.
\end{axi}
\begin{axi}[ AXIOMA DE LINEALIDAD]
	Dadas  dos funciones cualesquiera  $ f ,  g \in H$, y los números complejos $a$ y $ b, af +bg \in H. $
\end{axi}
\begin{axi}[ AXIOMA MÉTRICO]
	Para $\forall f, g, h \in H$, con la definición dada en (3) para el producto escalar son inmediatas las relaciones:
	\[ (f, g) = (\overline{g, f}) \]
	\begin{equation}
	(af, g) = a(f, g)
	\end{equation}
	\[ (f, f) = \int_{a}^{b}|f|^{2}dx \]
\end{axi}
\begin{axi}[ DE CONVERGENCIA] 

	a) Criterio de Cauchy
	$ \forall g_{n}, g_{m} \in H $
	\begin{equation}
	\parallel g_{n} - g_{n+p} \parallel < \varepsilon
	\end{equation}
	existe un elemento $ g \in H  $tal que:
	\begin{equation}
	\lim\limits_{n \rightarrow \infty} \parallel g - g_{n} \parallel < \varepsilon
	\end{equation}
	b) Separabilidad\\
	Se puede elegir una sucesión de funciones pertenecientes a $ H $, de forma que si $ g_{n} \in H $
	todo $ f \in H $ puede expresarse con error arbitrariamente pequeño.
\end{axi}
\section{REPRESENTACIÓN POR FUNCIONES ORTOGONALES}

Si$  f \in H $ dada la sucesión de funciones (1), cumpliendo la (2) de 2.6 podemos
representar la combinación lineal:
\setcounter{equation}{0}%reiniciar numeracion de las ecuaciones
\begin{equation}
g_{n} = a_{0}f_{0}+a_{1}f_{1}+\ldots+a_{n}f_{n}
\end{equation}
\begin{equation}
a_{k} = \int_{a}^{b}f.\overline{f}_{k}dx
\end{equation}
deducida de la distancia mínima de $ f $ a la combinación (1) y de forma semejante como hemos deducido en 2.5.\\
La distancia f a la combinación lineal (1) cuando sea menor que \underline{\underline{$\varepsilon$}} debe suceder:
\begin{equation}
\parallel f - g \parallel ^{2} = \int_{a}^{b} |f - g|^{2}dx < \varepsilon
\end{equation}
y en este caso escribiremos:
\begin{equation}
\text{  f   }  ^{\underline{\underline{2}}} \sum_{k = 1}^{n} akfk = g_{n}
\end{equation}
Si $ f(x) $ es una función de cuadrado integrable, el error disminuye al agregar términos en (4) y cuando la distancia tiende a cero estamos en el concepto b) del Axioma (4) de convergencia de 2.6 respecto a ser el espacio funcional separable.

\section{CONVERGENCIA UNIFORME Y CUADRÁTICA}

La convergencia uniforme de una función $ f(x) $ aproximada por una combinación lineal (1) de 2.7 para $ x \in (a, b) $ es:
\setcounter{equation}{0}%reiniciar numeracion de las ecuaciones
\begin{equation}
		|f(x)| - g(x)  < \varepsilon
\end{equation}
aproximándose uniformemente a $ f(x) $ en ei intervalo $ (a, b) $.\\
La convergencia uniforme de  $ g(x) \rightarrow f(x) $ en $ (a, b) $ implica la convergencia  cuadrática.\\
\begin{equation}
f(x)  =^{2} g(x) 
\end{equation}
porque la distancia de$f(x)$ a $g(x)$ según (3) de 2.7 será:
\begin{equation}
\int_{a}^{b}|f(x)-g(x)|^{2}dx < \varepsilon^{2}(b-a) < \varepsilon_{1}
\end{equation}
recordando la (1); luego:
\begin{equation}
g(x)^{2} = f(x)
\end{equation}
\section{FUNCIONES ORTONORMALIZADAS DE FOURIER}
\setcounter{equation}{0}%reiniciar numeracion de las ecuaciones
La sucesión de funciones (1) de 2.6 dentro del campo -a, +a, puede ser:
\[ f_{0}(x)=\frac{1}{\sqrt{2a}} \]
\begin{equation}
f_{1}(x)=\frac{1}{\sqrt{a}}\cos{\frac{\pi x}{a}}, \ldots, f_{2n-1}(x)=\frac{1}{\sqrt{a}}\cos{\frac{n \pi x}{a}}contenidos...
\end{equation}
\[ f_{2}(x)=\frac{1}{\sqrt{a}}sen{\frac{\pi x}{a}}, \ldots, f_{2n-1}(x)=\frac{1}{\sqrt{a}}sen{\frac{n \pi x}{a}} \]

Esta sucesion cie funciones se comprueba sencillamente que es una base ortonormalizada de un espacio  funcional por las propiedades de las integrales de senos y cosenos.\\

La representación de un elemento $ f \in H $ POR el conjunto de sucesiones ortonormalizadas de Fourier representando $ a'_{n} $ y $ b'_{n} $ las coordenadas en este espacio es:
\[ f(x) = \frac{a'_{0}}{\sqrt{2a}}+\frac{1}{\sqrt{a}}[a'_{1} \cos \frac{\pi x}{a}+b'_{1} sen \frac{\pi x}{a}]+ \]
\begin{equation}
+\ldots+\frac{1}{\sqrt{a}}[a'_{n}sen \frac{n \pi x}{a}+b'_{n}sen \frac{n \pi x}{a}]+\ldots+(n = 1, 2, 3 \ldots)
\end{equation}
Por la (2) de 2.7, tenemos:
\begin{equation}
	\left[  f(x), \frac{1}{\sqrt{2a}} \right]	= a'_{0} = \frac{1}{\sqrt{2a}} \int_{a}^{b} f(x)dx
\end{equation}
\[ a'_{1} = \left[ f(x), \frac{\cos \frac{\pi x}{a}}{\sqrt{a}} \right] = \int_{-a}^{a}\frac{f(x)}{\sqrt{a}} \cos \frac{\pi x}{a}dx = \]

\[ a'_{1} = \frac{1}{\sqrt{a}} \int_{-a}^{a} f(x)  \cos \frac{\pi x}{a}dx \]
\[ b'_{1} = \left[f(x), \frac{sen \frac{\pi x}{a}}{\sqrt{a}}\right] = \frac{1}{\sqrt{a}} \int_{-a}^{a} f(x) sen \frac{\pi x}{a} dx\]

\begin{subequations}
	\begin{equation}
	a'_{n} = \frac{1}{\sqrt{a}} \int_{-a}^{b} f(x) \cos \frac{n \pi x}{a}dx
	\end{equation}
		\[ (n = 1, 2, 3 \ldots) \]
	\begin{equation}
	b'_{n} = \frac{1}{\sqrt{a}} \int_{-a}^{a} f(x) sen \frac{n \pi x}{a}dx
	\end{equation}
\end{subequations}

\section{FUNCIÓN CARACTERÍSTICA}
A partir de la (2) en 2.9  podemos deducir la función característica en relación con la
función de densidad. Sustituyendo en (2) las coordenadas (4a) y (4b) tenemos:

